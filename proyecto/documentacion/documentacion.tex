\documentclass[a4paper,12pt]{article}
\usepackage[utf8]{inputenc}
\usepackage[T1]{fontenc}
\usepackage{lmodern} % Tipografía moderna
\usepackage{geometry}
\geometry{margin=1in}
\usepackage{amsmath}
\usepackage{graphicx}
\usepackage{setspace}
\usepackage{titlesec}
\usepackage{listings}
\usepackage{color}
\usepackage{xcolor}
\usepackage{fancyhdr}
\usepackage{tcolorbox}
\usepackage{hyperref}
\usepackage{tikz}

% Estilo de portada
\newtcolorbox{portadabox}[2][]{colback=white!5!gray, colframe=black,
title=#2, fonttitle=\bfseries, #1}

% Configuración de encabezados
\pagestyle{fancy}
\fancyhf{}
\fancyhead[L]{Proyecto de Programación: Calculadora de Nómina}
\fancyhead[R]{2024}
\fancyfoot[C]{\thepage}

% Configuración de estilo de código
\lstset{
    language=Python,
    basicstyle=\ttfamily\small,
    keywordstyle=\color{blue}\bfseries,
    commentstyle=\color{green!50!black},
    stringstyle=\color{red},
    numberstyle=\tiny\color{gray},
    numbers=left,
    stepnumber=1,
    numbersep=8pt,
    backgroundcolor=\color{white},
    frame=single,
    breaklines=true,
    breakatwhitespace=true,
    captionpos=b,
    showspaces=false,
    showstringspaces=false,
    tabsize=4
}

% Estilo de títulos
\titleformat{\section}{\normalfont\Large\bfseries}{\thesection}{1em}{}
\titleformat{\subsection}{\normalfont\large\bfseries}{\thesubsection}{1em}{}
\titleformat{\subsubsection}{\normalfont\normalsize\bfseries}{\thesubsubsection}{1em}{}

% Divisores decorativos
\newcommand{\divider}{
    \begin{center}
        \tikz{\draw[thick, color=gray] (0,0) -- (15,0);}
    \end{center}
}

% Configuración de hipervínculos
\hypersetup{
    colorlinks=true,
    linkcolor=blue,
    filecolor=magenta,
    urlcolor=cyan,
    pdftitle={Calculadora de Nómina},
    pdfpagemode=FullScreen,
}

\begin{document}

% Portada
\begin{titlepage}
    \begin{center}
        \vspace*{1cm}
        
        % Borde decorativo
        \begin{portadabox}[width=\textwidth]{}
            \begin{center}
                \includegraphics[width=0.3\textwidth]{Logo_URosario.png} % Logo de la universidad
                \vspace{1cm}
                
                \Huge\textbf{Proyecto Programación}\\[1cm]
                \Large\textbf{Calculadora de Nómina}\\[2cm]
                \Large\textbf{Documentación}\\[2cm]
                \Large\textbf{2024}\\[2cm]
            \end{center}
        \end{portadabox}
    \end{center}
\end{titlepage}

% Roles de los integrantes
\section*{Roles de los Integrantes}
\begin{itemize}
    \item \textbf{Thomas Ayala}: Gerente.
    \item \textbf{Felipe Sierra}: Diseño.
    \item \textbf{Valentina Mesa}: Documentación y pruebas.
\end{itemize}

\divider

\tableofcontents
\newpage

% Introducción
\section{Introducción}

El proyecto \textbf{Calculadora de Nómina} es una herramienta tecnológica backend diseñada para emprendimientos que necesiten gestionar nóminas de manera eficiente. Este sistema permite:
\begin{itemize}
    \item Ingresar datos de empleados y calcular sus salarios.
    \item Manejar la gestión de contraseñas para garantizar la seguridad.
    \item Registrar datos en un archivo CSV para futuras consultas.
    \item Ofrecer una interfaz gráfica basada en \texttt{pygame}.
\end{itemize}

El código del proyecto está alojado en el repositorio oficial: \url{https://github.com/Vale0911/proyecto_final}.

\divider

% Requisitos
\section{Requisitos}

\begin{itemize}
    \item \textbf{Sistema operativo}: Windows, MacOS o Linux.
    \item \textbf{Librerías necesarias}:
    \begin{itemize}
        \item \texttt{pygame}
        \item \texttt{flask}
    \end{itemize}
    \item \textbf{Resolución recomendada para Pygame}: 800x600 px.
    \item \textbf{Instalación}: Asegúrate de ejecutar:
    \begin{lstlisting}
    pip install -r requirements.txt
    \end{lstlisting}
\end{itemize}

\divider

% Manual de Usuario
\section{Manual de Usuario}

\begin{enumerate}
    \item Se debe tener instalado Python
    \item  Copiar el link de la terminal.
    \item Abre un navegador y accede a \url{http://localhost:5000}. (Ejemplo de como se ve.)
    \item Ingresa los datos requeridos: (Pygame)
    \begin{itemize}
        \item \textbf{ID del empleado}.
        \item \textbf{Horas trabajadas}.
        \item \textbf{Horas extras}.
    \end{itemize}
    \item Haz clic en "Calcular" para obtener el salario total.
    \item Los datos serán almacenados en formato CSV para futuras consultas.
\end{enumerate}

\divider

% Guía de Código
\section{Guía Interna del Código}

\subsection{Resumen de las funciones principales}
\begin{itemize}
    \item \textbf{Versión 2:}Mejora la interfaz utilizando Flask para conectar el backend con el frontend de HTML y CSS.
    \item \textbf{Versión 1:}Utiliza Pygame para la interfaz gráfica.
    
\end{itemize}

\subsection{Estructura del código}
\textbf{Arquitectura general:}Cliente-servidor.


\textbf{Organización de los Archivos y Carpetas del Proyecto:}
\begin{itemize}
    \item \textbf{proyecto/version1/:}Contiene la implementación basada en Pygame.
    \item \textbf{proyecto/version2/:}Contiene la implementación mejorada con Flask.
\end{itemize}
\textbf{Archivos principales:}
\begin{itemize}
    \item\textbf{proyecto/version1/Backend/calc.py:} Código de la versión 1.
    \item \textbf{proyecto/version2/main.py:} Código de la versión 2.
    \item \textbf{proyecto/version2/templates/inicio.html:}Página de inicio.
    \item \textbf{proyecto/version2/templates/login.html:} Página de login.
    
\end{itemize}

\subsection{Tecnologías utilizadas}
\begin{enumerate}
    \item \textbf{Lenguajes de Programación:}
    \begin{itemize}
            \item \texttt{Python} 
            \item \texttt{HTML}
            \item \texttt{CSS}
        \end{itemize}
    \item \textbf{Frameworks y Librerías:}
    \begin{itemize}
        \item \textbf{Versión 1:}\texttt{Pygame}
        \item \textbf{Versión 2:}\texttt{Flask}
    \end{itemize}
    \item \textbf{Requisitos del Entorno:}
    \begin{itemize}
        \item \texttt{Python 3.x}
        \item \texttt{Flask}
        \item \texttt{Pygame}
    \end{itemize}
\end{enumerate}

\subsection{Componentes clave}

\textbf{BaseDeDatos:} Clase para manejar la base de datos de usuarios. Este módulo gestiona las operaciones básicas de almacenamiento, recuperación y validación de usuarios mediante un archivo JSON. Permite agregar nuevos usuarios y validar credenciales existentes.

\begin{lstlisting}[language=Python, caption={Clase BaseDeDatos}]
class BaseDeDatos:
    def __init__(self):
        self.archivo = 'usuarios.json'
        self.usuarios = self.cargar_usuarios()

    def cargar_usuarios(self):
        if os.path.exists(self.archivo):
            with open(self.archivo, 'r') as f:
                return json.load(f)
        return {}

    def guardar_usuarios(self):
        with open(self.archivo, 'w') as f:
            json.dump(self.usuarios, f)

    def agregar_usuario(self, usuario, contrasena):
        if usuario in self.usuarios:
            return False
        self.usuarios[usuario] = contrasena
        self.guardar_usuarios()
        return True

    def validar_usuario(self, usuario, contrasena):
        return self.usuarios.get(usuario) == contrasena
\end{lstlisting}

\textbf{main.py:} Archivo principal que inicializa y ejecuta la aplicación Flask. Este archivo gestiona las rutas y las vistas principales de la aplicación, sirviendo como punto de entrada para el servidor Flask. Permite renderizar las páginas de inicio y login.

\begin{lstlisting}[language=Python, caption={Archivo main.py}]
from flask import Flask, render_template, request, redirect, url_for, session

app = Flask(__name__)

@app.route("/")
def inicio():
    return render_template("inicio.html")

@app.route("/login", methods=["GET", "POST"])
def login():
    # Lógica de autenticación
    pass

if __name__ == '__main__':
    app.run(debug=True)
\end{lstlisting}

\subsection{Interacciones Entre Módulos}

\begin{itemize}
    \item En la versión 2, \textbf{Flask} actúa como intermediario entre el usuario y los componentes del backend.
    \item La clase \texttt{BaseDeDatos} se utiliza para manejar la lógica de persistencia de datos, mientras que las plantillas HTML renderizan el contenido dinámico para el usuario.
\end{itemize}

\subsection{Funciones Representativas de la Interfaz Pygame}

En la \textbf{versión 1}, se emplean funciones para manejar eventos y presentar botones en pantalla. Ejemplo de la página de inicio:

\begin{lstlisting}[language=Python, caption={Función manejar\_pagina\_inicio}]
def manejar_pagina_inicio():
    pantalla.fill(BLANCO)
    mostrar_texto("Gestión de Usuarios y Nómina", 240, 50, NEGRO)
    boton_crear_usuario = crear_boton("Crear Usuario", 300, 200, 200, 50, AZUL, BLANCO)
    boton_iniciar_sesion = crear_boton("Iniciar Sesión", 300, 300, 200, 50, VERDE, BLANCO)
    return {"crear_usuario": boton_crear_usuario, "iniciar_sesion": boton_iniciar_sesion}
\end{lstlisting}

\subsection{Funciones Importantes para la Nómina}

\textbf{1. Agregar Usuario:} Esta función permite registrar nuevos usuarios en un archivo CSV, asegurando la persistencia de datos.

\begin{lstlisting}[language=Python, caption={Función agregar\_usuario}]
def agregar_usuario(email, password_hash):
    with open('usuarios.csv', mode='a', newline='') as archivo:
        escritor = csv.writer(archivo)
        escritor.writerow([email, password_hash])
\end{lstlisting}

\textbf{2. Validación de Usuarios:} Método en la clase \texttt{BaseDeDatos} que compara credenciales con los datos existentes.

\begin{lstlisting}[language=Python, caption={Método validar\_usuario}]
def validar_usuario(self, usuario, contrasena):
    return self.usuarios.get(usuario) == contrasena
\end{lstlisting}

\textbf{3. Cálculo de Nómina:} Aunque solo es un esquema inicial, este método prevé realizar cálculos automáticos de salarios y deducciones.

\begin{lstlisting}[language=Python, caption={Esquema inicial para el cálculo de nómina}]
def calculador_nomina():
    # Configuración inicial
    # Captura interactiva
    # Validar entrada de datos
    # Buscar al empleado
    # Mostrar resultados
    pass
\end{lstlisting}

% Comparación de Avances
\section{Comparación de Avances Semanales}

\subsection*{Semana 1: Configuración inicial}
\begin{itemize}
    \item Configuración básica de Pygame y diseño HTML/CSS inicial.
    \item Problema: Cuadros de texto no alineados correctamente.
    \begin{lstlisting}[language=Python]
    
import pygame
pygame.init()

c_base = (0,0,0) #negro
c_rectangulo = (169, 169, 169)  # gris
c_texto = (255, 255, 255) # blanco
Largo = 1000  #lagro y acnho de la venta
Ancho = 700

screen = pygame.display.set_mode((Largo, Ancho)) # define screen y ademas define el largo y anho de la ventana
pygame.display.set_caption('Calculador de nomina') # literalmente solo nombra la ventaa xd\
fuente = pygame.font.SysFont(None, 40) #fuente para e texto creo que es arial segun reddit no lo es pero no onfio en esos locos]


def texto_id():
    
    rect_x = 100 #posicion x del rectangulo
    rect_y = 100 #posicion y del rectangulo
    rect_ancho = 300 #tamano del rectangulo
    rect_alto = 50 #ancho del rectangulo

    pygame.draw.rect(screen, c_rectangulo, (rect_x, rect_y, rect_ancho, rect_alto)) #literalmente solo dibuje
    
    texto = fuente.render("Id del Empleado", True, c_texto) #que texto quiero que salga en el recatangulo

    screen.blit(texto, (rect_x + 50, rect_y +10)) #margenes del texto

def texto_horas():
    
    rect_x = 100 
    rect_y = 200 
    rect_ancho = 300 
    rect_alto = 50 #

    pygame.draw.rect(screen, c_rectangulo, (rect_x, rect_y, rect_ancho, rect_alto)) 
    
    texto = fuente.render("horas trabajdas " , True, c_texto) 

    screen.blit(texto, (rect_x + 40, rect_y +10)) 

def texto_extas():
    
    rect_x = 100 
    rect_y = 300 
    rect_ancho = 300 
    rect_alto = 50 #

    pygame.draw.rect(screen, c_rectangulo, (rect_x, rect_y, rect_ancho, rect_alto)) 
    
    texto = fuente.render("horas extras " , True, c_texto) 

    screen.blit(texto, (rect_x + 70, rect_y +10))

def main():
    running = True # para que no se cierre la ventana
    while running:
        screen.fill(c_base) #darle un color a la ventana
        texto_id()
        texto_horas()
        texto_extas()
        dibujar_input_rects()
        pygame.display.flip() #no se ccierre la ventana
        
    pygame.quit() #cerrar la ventana

if __name__ == "__main__": #correr el codigo NO TOCAR (no se como funciona)
    main()
    \end{lstlisting}
\end{itemize}

\subsection*{Semana 2: Avances en interactividad}
\begin{itemize}
    \item Implementación de cuadros de entrada interactivos.
    \item Problema: Manejo inadecuado de eventos para entrada de texto.

    \begin{lstlisting}[language=python]
        
import pygame
pygame.init()

c_base = (0,0,0) #negro
c_rectangulo = (169, 169, 169)  # gris
c_texto = (255, 255, 255) # blanco
Largo = 1000  #lagro y acnho de la venta
Ancho = 700

input_rects = [pygame.Rect(600, 100, 300, 50), pygame.Rect(600, 200, 300, 50), pygame.Rect(600, 300, 300, 50)]
input_textos = ["", "", "", ]  # Lista de textos para cada cuadro
input_activo = [False, False, False,]  # Lista de estados de activación para cada cuadro


screen = pygame.display.set_mode((Largo, Ancho)) # define screen y ademas define el largo y anho de la ventana
pygame.display.set_caption('Calculador de nomina') # literalmente solo nombra la ventaa xd\
fuente = pygame.font.SysFont(None, 40) #fuente para e texto creo que es arial segun reddit no lo es pero no onfio en esos locos]


def texto_id():
    
    rect_x = 100 #posicion x del rectangulo
    rect_y = 100 #posicion y del rectangulo
    rect_ancho = 300 #tamano del rectangulo
    rect_alto = 50 #ancho del rectangulo

    pygame.draw.rect(screen, c_rectangulo, (rect_x, rect_y, rect_ancho, rect_alto)) #literalmente solo dibuje
    
    texto = fuente.render("Id del Empleado", True, c_texto) #que texto quiero que salga en el recatangulo

    screen.blit(texto, (rect_x + 50, rect_y +10)) #margenes del texto

def texto_horas():
    
    rect_x = 100 
    rect_y = 200 
    rect_ancho = 300 
    rect_alto = 50 #

    pygame.draw.rect(screen, c_rectangulo, (rect_x, rect_y, rect_ancho, rect_alto)) 
    
    texto = fuente.render("horas trabajdas " , True, c_texto) 

    screen.blit(texto, (rect_x + 40, rect_y +10)) 

def texto_extas():
    
    rect_x = 100 
    rect_y = 300 
    rect_ancho = 300 
    rect_alto = 50 #

    pygame.draw.rect(screen, c_rectangulo, (rect_x, rect_y, rect_ancho, rect_alto)) 
    
    texto = fuente.render("horas extras " , True, c_texto) 

    screen.blit(texto, (rect_x + 70, rect_y +10))

def dibujar_input_rects():
    for i, rect in enumerate(input_rects):
        pygame.draw.rect(screen, c_texto, rect)
        texto = fuente.render(input_textos[i], True, c_base)
        screen.blit(texto, (rect.x + 40, rect.y + 10))


def main():
    running = True # para que no se cierre la ventana
    while running:
        for event in pygame.event.get():
            if event.type == pygame.QUIT:
                running = False
            if event.type == pygame.MOUSEBUTTONDOWN:
                for i, rect in enumerate(input_rects):
                    if rect.collidepoint(event.pos):
                        input_activo = [False] * len(input_rects)  # Desactivar todo xd
                        input_activo[i] = True  # Activar el input seleccionado
                        break

            if event.type == pygame.KEYDOWN:
                for i, activo in enumerate(input_activo):
                    if activo:  # para que el imput solo se deje cuando uno le de click
                        if event.key == pygame.K_BACKSPACE: #boton de borado sirve graias a dios
                            input_textos[i] = input_textos[i][:-1]
                        elif event.key == pygame.K_RETURN:
                            print(f"Texto ingresado en cuadro {i+1}: {input_textos[i]}") # prueba por que no confio en mi mismo
                            input_activo[i] = False  # dejar de esribir cunado se le de ente al escribri
                        else:
                            input_textos[i] += event.unicode #caracteres especiales

        screen.fill(c_base) #darle un color a la ventana
        texto_id()
        texto_horas()
        texto_extas()
        dibujar_input_rects()
        pygame.display.flip() #no se ccierre la ventana
        
    pygame.quit() #cerrar la ventana


if __name__ == "__main__": #correr el codigo NO TOCAR (no se como funciona)
    main()
    \end{lstlisting}
\end{itemize}

\subsection*{Semana 3: Código final}
\begin{itemize}
    \item Integración de backend, frontend y base de datos.
    \item Implementación funcional de manejo de nóminas.
\end{itemize}

\divider

% Conclusión
\section{Conclusión}

El proyecto \textbf{Calculadora de Nómina} permitió combinar habilidades técnicas en Python y diseño web para construir una solución funcional y robusta. A través de iteraciones semanales, se resolvieron problemas clave y se logró integrar todas las funcionalidades necesarias para una experiencia de usuario fluida.

\end{document}

